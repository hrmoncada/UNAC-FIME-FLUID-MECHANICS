\documentclass[10pt,a4paper]{article}
\usepackage{amsmath}
\usepackage[margin=1in]{geometry}
\usepackage{amsmath}
\usepackage{amssymb}
\usepackage{color}
\usepackage{graphicx}
\usepackage{fancyhdr}
\usepackage{url}
%%%%%%%%%%%%%%%%%%%%%%%%%%%%%%%%%%%%%%%%%%%%%%%%%%%%%%
% lstlisting
%%%%%%%%%%%%%%%%%%%%%%%%%%%%%%%%%%%%%%%%%%%%%%%%%%%%%%
\usepackage{listings}             % Include the listings-package
\usepackage{alltt}
\definecolor{codegreen}{rgb}{0,0.6,0}
\definecolor{codegray}{rgb}{0.5,0.5,0.5}
\definecolor{codepurple}{rgb}{0.58,0,0.82}
\definecolor{backcolour}{rgb}{0.95,0.95,0.92}

\lstdefinestyle{mystyle}{
    backgroundcolor=\color{backcolour},
    commentstyle=\color{codegreen},
    keywordstyle=\color{codepurple},
    numberstyle=\tiny\color{codegray},
    stringstyle=\color{codepurple},
    basicstyle=\ttfamily\small,
    breakatwhitespace=false,
    breaklines=true,
    captionpos=b,
    keepspaces=true,
    numbers=left,
    numbersep=5pt,
    showspaces=false,
    showstringspaces=false,
    showtabs=false,
    tabsize=2,
    language=Fortran
}

\lstset{style=mystyle}

%%%%%%%%%%%%%%%%%%%%%%%%%%%%%%%%%%%%%%%%%%%%%%%%%%%%%%
% Graphs path to the Picture Folder
%%%%%%%%%%%%%%%%%%%%%%%%%%%%%%%%%%%%%%%%%%%%%%%%%%%%%%
\graphicspath{{Pictures/}{Data/}} % Two folders Picture and Data


\newcommand{\myUniversidad}{UNIVERSIDAD NACIONAL DEL CALLAO}
\newcommand{\myFacultad}{FACULTAD DE CIENCIAS NATURALES Y MATEM\'ATICA}
\newcommand{\myEscuelaProfesional}{ESCUELA PROFESIONAL DE F\'ISICA}
\newcommand{\myCurso}{F\'ISICA COMPUTACIONAL 2}
\newcommand{\myTarea }{Tarea }
\newcommand{\myNombre}{Tu nombre}
\newcommand{\myEmail}{email\_name@unac.gob,pe}
\newcommand{\myNumTarea}{1}

\pagestyle{fancyplain}
\lhead{\fancyplain{}{\textbf{Tarea \myNumTarea}}}      
\rhead{\fancyplain{}{\myNombre}}

%%%%%%%%%%%%%%%%%%%%%%%%%%%%%%%%%%%%%%%%%%%%%%%%%%%%%%
% bibliography
%%%%%%%%%%%%%%%%%%%%%%%%%%%%%%%%%%%%%%%%%%%%%%%%%%%%%%
%\usepackage{biblatex}
\bibliographystyle{plain}

%%%%%%%%%%%%%%%%%%%%%%%%%%%%%%%%%%%%%%%%%%%%%%%%%%%%%%
% Begin Document
%%%%%%%%%%%%%%%%%%%%%%%%%%%%%%%%%%%%%%%%%%%%%%%%%%%%%%
\begin{document}
%--------------------------------------------------------------------------------------------------
\thispagestyle{plain} % plain permite que la primera pagina no aparescan la barra en la parte superior de la pagina
%--------------------------------------------------------------------------------------------------
\begin{minipage}{.30\textwidth}
 \includegraphics[width=.5\textwidth]{Escudo_UNAC.png}\\[.5cm]
\end{minipage}
\begin{minipage}{.65\textwidth}
 \begin{center}  % Center the following lines
\Large{                
\myUniversidad\\
\myFacultad\\ 
\myCurso\\
{\myTarea \myNumTarea}\\
\myNombre\\
\myEmail\\
\today}
\end{center} 
\end{minipage}\\
\centerline{\underline{\hspace{7in}}}
%\maketitle

\section*{Instrucciones}
Resuelva cuidadosamente cada problema. Si sudas, es normal. Si lloras, también. Asegúrate de incluir todos los pasos, unidades, diagramas (si aplica), y una pizca de paciencia. Si algo no cuadra... probablemente sea culpa del universo (o de las unidades mal convertidas).
\vspace{-1em}
\begin{center}
\textbf{Fuente:} Física Universitaria, Sears y Zemansky, 13ª Edición, Vol. 1 (2013), Capítulo 12, pág. 394
\end{center}

\vspace{1em}

% --- Repite este bloque para cada problema ---
\section*{Problema 1}
\textbf{Enunciado:} \\
Escribe aquí el enunciado del problema (o una versión resumida si lo tienes en el libro).

\vspace{1em}
\textbf{Solución:} \\

Aquí va la resolución paso a paso del problema.

\begin{itemize}
  \item Paso 1: Identificar qué demonios está pasando.
  \item Paso 2: Aplicar la ecuación correspondiente.
  \item Paso 3: Resolver, mantener la calma, y verificar unidades.
  \item Paso 4: Escribir la respuesta final con sus respectivas unidades.
\end{itemize}

% Puedes incluir una figura si es necesario
\begin{figure}[h]
\centering
\includegraphics[width=1\textwidth,height=.45\textheight]{fig1.png}
\caption{Diagrama esquemático del problema.}
\end{figure}

% Ejemplo de segundo problema
\section*{Problema 2}
\textbf{Enunciado:} \\
Escribe aquí el enunciado del problema (o una versión resumida si lo tienes en el libro). \cite{Ding_and_Williams}.\cite{Yang_Kurth_Willians}
%      \\\\
% Escribe aquí el enunciado
% ... \\\\
\textbf{Solución:} \\
\begin{equation}
  x_{n+1}  =  x_n - m\;f(x)\left[\frac{x_n-x_{n-1}}{f(x_n)-f(x_{n-1}}\right]
\end{equation}
\hspace{1mm}\\
For $m=\frac{1+\sqrt{5}}{2}=1.618$\\
\begin{verbatim}
Steps        xa              xb               xi           f(xi)       |(xn+1-xn)|
----------------------------------------------------------------------------------
  0  1.500000000000  1.600000000000  1.633191538329 -0.001945949763  0.033191538329 
  1  1.600000000000  1.633191538329  1.564416498720 -0.000020351034  0.068775039609 
  2  1.633191538329  1.564416498720  1.563240412447 -0.000028545785  0.001176086273 
  3  1.564416498720  1.563240412447  1.569869184045 -0.000000429797  0.006628771598 
  4  1.563240412447  1.569869184045  1.570033141259 -0.000000291226  0.000163957214 
  5  1.569869184045  1.570033141259  1.570590682342 -0.000000021145  0.000557541083 
  6  1.570033141259  1.570590682342  1.570661309883 -0.000000009115  0.000070627541 
  7  1.570590682342  1.570661309883  1.570747894571 -0.000000001173  0.000086584688 
  8  1.570661309883  1.570747894571  1.570768583633 -0.000000000385  0.000020689062 
  9  1.570747894571  1.570768583633  1.570784932394 -0.000000000065  0.000016348761 
 10  1.570768583633  1.570784932394  1.570790299953 -0.000000000018  0.000005367559 
 11  1.570784932394  1.570790299953  1.570793673498 -0.000000000004  0.000003373545 
 12  1.570790299953  1.570793673498  1.570794985787 -0.000000000001  0.000001312289 
 13  1.570793673498  1.570794985787  1.570795714281 -0.000000000000  0.000000728494 
\end{verbatim}
\begin{lstlisting}
! Codigo Fortran aqui
PROGRAM ejemplo
  IMPLICIT NONE
  INTEGER :: i, n
  REAL :: x, sum

  n = 100
  sum = 0.0

  DO i = 1, n
    x = REAL(i)
    sum = sum + x**2
  END DO

  PRINT *, 'La suma de los cuadrados es: ', sum

END PROGRAM ejemplo
\end{lstlisting}
%      \\\\
% Escribe aquí tu soluci\'on paso a paso. Puedes incluir ecuaciones, figuras, explicaciones y c\'odigo.
% ... \\\\

% ...y así sucesivamente hasta el problema 20


\begin{figure}[h]
\centering
\includegraphics[width=1\textwidth,height=.45\textheight]{fig2.png}
\caption{Diagrama esquemático del problema.}
\end{figure}

%%%%%%%%%%%%%%%%%%%%%%%%%%%%%%%%%%%%%%%%%%%%%%%%%%%%%%%%%%%%%%%%%%%%%%%%%%%%%%%%
%                                   Appendices page                            %                    
%%%%%%%%%%%%%%%%%%%%%%%%%%%%%%%%%%%%%%%%%%%%%%%%%%%%%%%%%%%%%%%%%%%%%%%%%%%%%%%%
\appendix
\addcontentsline{toc}{section}{Appendices}
\section*{Appendices}
\input{Appendix.tex}
%%%%%%%%%%%%%%%%%%%%%%%%%%%%%%%%%%%%%%%%%%%%%%%%%%%%%%%%%%%%%%%%%%%%%%%%%%%%%%%%
%                  Bibliography  or References page                            %                    
%%%%%%%%%%%%%%%%%%%%%%%%%%%%%%%%%%%%%%%%%%%%%%%%%%%%%%%%%%%%%%%%%%%%%%%%%%%%%%%%      
% \newpage
%\nocite{*}
\bibliography{References/myBibliography}
\end{document}
