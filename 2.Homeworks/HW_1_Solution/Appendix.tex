\appendix
%\doublespacing
%\chapter{Appendix}

\section{Front-end and Back-end}
In software engineering, the terms \verb+front-end+ and \verb+back-end+ refer to the separation
of concerns between the presentation layer (\verb+front-end+), and the data access layer (\verb+back-end+) of a piece of software, 
or the physical infrastructure or hardware. Therefore in a HPC, 
\begin{itemize}
\item The login servers are called \verb+front-ends+ because you do not run your calculations there.
\item Rather run your calculations on \verb+back-end+ compute servers.
\item The \verb+front-end+ server provides access to compute servers via the \verb+batch+ system, using the \verb+qsub+ command.
\end{itemize}

\section{Hard Link or Symbolic Link}
The \verb+ln+ command is a \verb+Linux/Unix+ command used to create file links to an existing file. 
\subsection{Link types}
\begin{itemize}
\item There are two types of links
\begin{enumerate}
\item  \textbf{hard links:} Refer to the specific location of physical data.
A hard link allows multiple filenames to be associated with the same file since a hard link points to the 
inode of a given file, the data of which is stored on disk.
\item  \textbf{symbolic links:} Refer to a symbolic path indicating the abstract location of another file.
A symbolic links are special files that refer to other files by name.
\end{enumerate}
\item The \verb+ln+ command by default creates hard links, and when called with the command line parameter \verb+ln -s+
creates symbolic links.
\item Most operating systems prevent hard links to directories from being created since such a capability could disrupt
the structure of a file system and interfere with the operation of other utilities. 
\item The \verb+ln+ command can however be used to create symbolic links to  non-existent files. 
\end{itemize}

\subsection{Examples}
\begin{enumerate}
 \item Example 1
\begin{lstlisting}[language=bash,numbers=none] 
$ ln -s source_file target_file
\end{lstlisting}
\begin{lstlisting}[language=bash,numbers=none] 
$ ls -l source_file target_file
-rw-r--r--  1 veryv  wheel  0 Mar  7 22:01 source_file
lrwxr-xr-x  1 veryv  wheel  5 Mar  7 22:01 target_file -> source_file
\end{lstlisting} 
\item Example 2 - Create a symbolic link for \verb+/home/Desktop/Links/Example/example.cpp+ as \verb+/home/Test/example.cpp+,
copy paste the following command
\begin{lstlisting}[language=bash,numbers=none] 
$ ln -s   /home/Desktop/Links/Example/example.cpp    /home/Test/example.cpp
\end{lstlisting}
\begin{lstlisting}[language=bash,numbers=none] 
$ ll
lrwxrwxrwx 1 vivek  vivek    16 2007-09-25 22:53 example.cpp -> /home/Desktop/Links/Example/example.cpp
\end{lstlisting}
\end{enumerate}

\section{Securely Copy (SCP) Files}
\verb+SCP+ allows files to be copied to, from, or between different hosts (between a local host and a remote host or between two remote hosts.).
It uses \verb+ssh+ for data transfer and provides the same authentication and same level of security as \verb+ssh+.
\begin{enumerate}
 \item Copy the file \verb+foobar.txt+ from a remote host to the local host
\begin{lstlisting}[language=bash,numbers=none] 
$ scp your_username@remotehost.edu:foobar.txt /some/local/directory 
\end{lstlisting}
\item How to \verb+scp+ a file to LANL-IC \verb+turquoise/+
\begin{lstlisting}[language=bash,numbers=none] 
$ scp filename username@wtrw.lanl.gov:username@gr-fe.lanl.gov:/remote/path/to/file
\end{lstlisting}
\item Also shorter probably works
\begin{lstlisting}[language=bash,numbers=none] 
scp filename wtrw:gr-fe:/remote/path/to/file
\end{lstlisting}
\end{enumerate}
